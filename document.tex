\documentclass[jou,a4paper,notxfonts]{apa}
\usepackage{graphicx} 
\usepackage{palatino}
\usepackage{fancyhdr}
\pagestyle{fancy}


% header on first page
\fancypagestyle{plain}{%
\fancyhf{} % clear all header and footer fields
\fancyhead[L]{\small
Journal of Eye Movement Research\\
1(1):1, 1-2}
\fancyfoot[C]{\thepage}
\renewcommand{\headrulewidth}{0pt}
\renewcommand{\footrulewidth}{0pt}}

\title{Interacting with Objects in the Environment by Gaze and Hand Gestures}

%please refer to http://www.ilsp.gr/homepages/protopapas/apacls.html#titlhead for different numbers of authors and affiliations
\threeauthors{First Author}{Second Author}{Third Author}
\threeaffiliations{First affiliation}{Second Affiliation}{Third Affiliation}


\journal{}
\volume{}

\abstract{
Head-mounted gaze trackers can be used for gaze estimation purposes in mobile scenarios. A wireless gaze tracker in the form of gaze tracking glasses permits the continuous monitoring of a subject's point of regard on the surrounding 3D environment. In this work, we combine gaze tracking and hand gesture recognition to allow a subject to interact with objects in the environment by gazing at them (pointing), and controlling the object using hand gesture commands. A set of gaze tracking glasses is made from low-cost hardware which consists of glasses' frame, eye tracking camera and the scene camera. An open source gaze estimation software is used for eye tracking and obtaining the user's gaze data. A binary beacon recognition library is used to identify objects in the environment. A hand gesture classification algorithm is used to recognize hand-based control commands. When combining all these elements, the emerging system permits a subject to move freely in an environment, select the object he wants to interact with by gaze, and transmit a control command to it by performing a hand gesture. 
%specific object is identified by the binary beacon visible in its surface. 
 This innovative HCI paradigm opens up new forms of interaction with objects in an smart environment.
\linebreak
\linebreak
{\bf Keywords: Keywords: Gaze Tracking, Head-Mounted Gaze Tracker, Mobile Interaction, Gaze Interaction, HCI}}

\acknowledgements{}
\shorttitle{Interacting with Objects in the Environment by Gaze and Hand Gestures}
\rightheader{}

% repeat the authors here (use et al if more than three authors):
\leftheader{ }

\begin{document}
\maketitle

\lhead{\small
Journal of Eye Movement Research\\
1(1):1, 1-2
}
\rhead{
Author, A., Author, B., \& Author, C. (2007)\\
Short Title
}
\thispagestyle{plain}

\section{Introduction} 
% We should clearify that this paper does not contribute in the hand gesture recognition but it uses the gesture and gaze in a mobile situation for controlling a robot.
Body language and gaze are important forms of communication among humans. In this work we present a system that combines gaze pointing and hand gestures to interact with objects in the environment. Our system merges a video-based gaze tracker, a hand gesture classifier and a visual marker recognition module into an innovate HCI device that permits novel forms of interaction with electronic devices in the environment. Gaze is used as a pointing mechanism to select the object with which the subject wants to interact. A visual binary marker attached to the object is used for identification of the object by the system. Finally, a hand gesture is mapped to a specific control command that makes the object carry out a particular function.

Using gaze for interaction with computers was initiated in the early 1980’s [Bolt, 1982] and further developed by [Ware 1987]. Today, gaze interaction is mostly done using a remote eye tracker with a single user sitting in front of a computer display. However, the head-mounted gaze trackers (HMGT) allow for a higher degree of mobility and flexibility, where the eye tracker is mounted on the user and thus allows gaze to be estimated when e.g. walking and driving. HMGT systems are commonly used for estimating the gaze point of the user in his field of view. However, the point of regard (PoR) obtained by head-mounted gaze trackers can be used for interaction with many different types of objects present in our daily activities in 3D environments. There have been some work done on using gaze for interaction with computers in the mobile scenarios using head-mounted gaze trackers [Mardanbegi]. Despite the fact that gaze can be used as a mechanism for pointing in many interactive applications, eye information has been shown to be limited for inter- action. Point of regard can be used for pointing, but not for yielding any additional commands. The main reason is that it is unnatural to overload a perceptual channel such as vision with a motor control task [Zhai, Morimoto, Ihde 1999]. Therefore, other interaction modalities such as body gestures and speech together with gaze can be used for enhancing the gaze-based interaction with computer and also with the environment. 
%the last sentence above should be corrected
 In this paper, we use the hand gestures for enhancing the gaze interaction in a fully mobile scenario . 
 
 Automatic gesture recognition is a topic in computer science and language technology that strives to interpret human gestures via computational algorithms. Gestures can originate from any bodily motion or state but commonly originate from the face or hands. An appealing feature of gestural interfaces is that they make it possible for users to communicate with objects without the need for external control devices. Hand gestures are an obvious choice as a mechanism to interact with objects in the environment \cite{Rozado2012b,myicann2010}. Automated hand gesture recognition is challenging since in order for such an approach to represent a serious alternative to conventional input devices, applications based on computer vision should be able to work successfully under uncontrolled light conditions, backgrounds and perspectives. In addition, deformable and articulated objects like hands represent added difficulty both for segmentation and shape recognition purposes. This paper dose not contribute in the hand gesture recognition, but suggests the combination of gaze and hand gestures as an alternative to the conventional methods that are used for gaze interaction such as: blinking (e.g., [MacKenzie 2008]), dwelling (e.g., [Jacob 1990]), and gaze gesture (e.g., [isokosi 2000]). We use the scene image of the HMGT system for recognizing the hand gestures and for recognizing the visual markers attached to the gazed objects. The hand gesture recognition module we developed here is able to detect a hand in front of the scene camera of the HMGT and the number of fingers that the hand is holding up. 
 
 In Summary, this work represents a proof of concept for an innovative form of interacting with objects in the environment by combining gaze and hand gestures. Interaction is achieved by gazing at an object in the environment and carrying out a hand gesture. The hand gesture specifies a certain command and gazing at the object, and the visual marker associated to it, make only that specific object to respond to the subsequent hand gesture. The off-the-shelf components used to build the hardware, the low cost of building the entire system and the open source nature of the algorithms used, make this form of interaction amenable for spreading among academic institutions and research labs to further investigate and stretch the possibilities of this innovative HCI paradigm.

The remaining paper is structured as follows. First...XXXX


% Diako: <the following paragraph about the eye tracking should be removed>
%Eye tracking refers to the monitoring of eye movements. Gaze tracking is the process of measuring the point of regard (where a person is looking). A head mounted eye tracker is a device that permits measuring eye movement while the subject is moving around an environment and estimate the position of gaze in the environment. Eye trackers are used in research on the visual system, in psychology, in cognitive linguistics, product design and in Human Computer Interaction
%(HCI) research \cite{myiwann2011}. There are a number of methods for measuring eye movement. The most popular variant uses video images from which the eye position is extracted. Other methods use search coils or are based on electrooculography.




\section{Related Work}
%Diako! My subsections are only a suggestion. you know better than me how to properly organize this section :)
%Related work on mobile gaze-based interaction, and hand gesture interaction with robot

Mardanbegi et. al [] presented a method for measuring the head gestures by the eye trackers. They applied their method for controlling the objects in the environment by gazing at objects and performing a head gesture. In their implementation for the head-mounted eye trackers they used the scene image for recognizing the objects and also to ensure that the PoR is on the object during the gesture. In contrast, in this paper, we have used the hand gestures and gaze as a multimodal choice for interaction. In this paper, we have used the information provided by the cameras of the head-mounted eye tracker for measuring the hand gestures. 


%There has been substantial research in hand/body gesture in for human-computer interaction. Lenman et al. explored the use of pie- and marking menus in hand gesture-based interaction[12].



%
%\subsection{Head mounted eye trackers}
%%Citing the papers from Diako on this subject
\subsection{Gestures in HCI}
%Please cite \cite{Rozado2012a}, \cite{mardanbegi2012eye} and \cite{Rozado2012} 
%
\subsection{Mobile Gaze Enhanced Interaction}

%the paragraphs below  should go to the implementation section 
 %We define the gesture as holding the hand with a preset number of fingers for a predefined dwell time of 1 second and moving it in a particular direction: up, down, left  or right. The hand gesture recognition worked well for natural skin color, but using a latex glove of a color not present in the environment improves performance.
%Visual marker recognition systems consist of a set of patterns that can be detected by a computer equipped with a camera and an appropriate detection algorithm \cite{middel19detection}. Markers placed in the environment provide easily detectable visual cues that can be associated to specific objects for identification purposes. In this work, we use the open source library XXXXXXXXXXXXXXXXX FOR DIAKO TO FILL XXXXXXXXXXXXXXXXXXXXXXXXXXXXXXXX. This library provides our system with the ability to recognize specific objects in the environment.
%We used an open source gaze tracker, described in \cite{Mardanbegi2011}, to monitor user's gaze. The gaze tracker provides real-time gaze estimation over the user?s field of view, as captured by the scene camera attached to the glasses,The system recognizes objects in the environment by detecting visual markers associated to them through the scene camera also referred to here as field of view camera. When the subject carrying the glasses gazes at an object, the visual marker placed nearby is recognized by the visual marker recognition library operating over the scene camera video stream. When a visual marker has been detected, a temporal window opens for detection of a hand gesture on the part of the user. To interact with an object in the environment, the user of the glasses needs to first look at the visual marker associated to the object that identifies the object and then perform the hand gesture. In this way, only that particular object in the environment will react to the hand gesture.


\section{Method}
%Jeremy! My subsections are only a suggestion. you know better than me how to properly organize this section :)

\subsection{Gaze Tracking}
Eyes are used by humans to obtain information about the surroundings and to communicate information. When something
attracts our attention, we position our gaze on it, thus performing a \textit{fixation}. A fixation usually has a
duration of at least 150 milliseconds (ms). The fast eye movements that occur between fixations are known as
\textit{saccades}, and they are used to reposition the eye so that the object of interest is projected onto the fovea.
The direction of gaze thus reflects the focus of \textit{attention} and also provides an indirect hint for
\textit{intention} \cite{velichkovsky}.


A video-based gaze tracking system seeks to find where a person is looking, i.e. the Point of Regard (PoR), using images
obtained from the eye by one or more cameras. Most systems employ infrared illumination that is invisible to the human
eye and hence it is not distracting for the user. Infrared light improves image contrast and produces a reflection on
the cornea, known as corneal reflection or glint. Eye features such as the corneal reflections and the center of the
pupil/iris can be used to estimate the PoR. Figure \ref{screenGazeTracker} shows a screenshot of an eye being tracked by
the open-source Haytham Gaze Tracker \cite{mardanbegi2012eye}. In this case, the center of the pupil and two
corneal reflections are the features being tracked.

%How to insert a figure (image):
\begin{figure}[tp]
 \fitbitmap{figures/screenGazeTracker.jpg}
 \caption{\textbf{The Open Source Haytham Gaze Tracker Tracking the Eye.} The
features tracked in the image are the pupil center and two corneal reflections. These features are 
used by the gaze estimation algorithms to determine the PoR of the user on the scene camera.}
 % the label is used to put refrences to this figure in the text
 \label{screenGazeTracker}
\end{figure}

Depending on the hardware configuration of the different components, gaze tracking systems can be classified as either
\textit{remote} or \textit{head-mounted}. In remote systems, the camera and the light sources are detached from the
user, normally around the device screen, whereas in head-mounted systems the components are placed on the user's head, usually
mounted on a helmet or a pair of safety glasses \cite{lowcostitugazetracker}.

Head-mounted eye trackers can be used for mobile interaction as well as gaze estimation purposes.
In this work, we have build a low-cost headmounted eye tracker using off-the-shelf components.  The system consists of
safety glasses, batteries, wireless eye camera and the wireless scene or field of view camera. The wireless eye camera
is equpped with infrared emitting diodes that permit the associated software to monitor the position of the pupil on the
image and the infrared reflection in the iris. The vector difference between the pupil center and the glint reflection
in the iris can be tracked during a calibration procedure to build a user specific model of his gaze. The calibration
procedure consists on the user looking at a number of points on the environment and marking them on the scene camera
video stream while the user fixates on them.  Once a calibration procedure is completed, the gaze estimation algorithm
is able to determining the point of regard of the user in the environment.


We use the open-source Haytham \cite{Mardanbegi2011} gaze tracker developed by Diako Mardanbegui at the IT University of
Copenhagen for mobile gaze estimation and extended it with a hand gesture recognition module that enables specific
interaction with objects in the environment through gaze and hand gestures.

\subsection{Constructing the eye tracking glasses}

Figure \ref{gazeTrackingGlasses}

\begin{figure}[tp]
 \fitbitmap{figures/gazeTrackingGlasses.jpg}
 \caption{\textbf{Low Cost Gaze Tracking Glasses.} Bla bla.}
 % the label is used to put refrences to this figure in the text
 \label{gazeTrackingGlasses}
\end{figure}


Figure \ref{jeremy}

\begin{figure}[tp]
 \fitbitmap{figures/jeremy.jpg}
 \caption{\textbf{Low Cost Gaze Tracking Glasses On a Subject.} Bla bla.}
 % the label is used to put refrences to this figure in the text
 \label{gazeTrackingGlasses}
\end{figure}


\subsection{Apparatus}
\subsubsection{Eye-tracking system}
\section{Results}


Materials:
Plastic safety glasses
Wireless camera with infra-red LEDs (eye camera)
Wireless camera	(scene camera)
Tin strips
Araldite
Steel wire
Tape
Double sided tape
9V battery
Tin snips

1. An area was traced onto the lens of the glasses where the eyes will be approximately when the user puts on the safety googles. 
2. Using the tin snips, the plastic part of the lenses bounded by the previously traced area was cut away. It is important that the
majority of the structure of the glasses is left intact to allow for stability.
3. Tin was cut to the size and shape of the infrared camera using the tin snips.
4. Steel wire was cut to a size of 25cm.
5. The steel wire was then attached to piece of tin using araldite.
6. Using double sided tape, the tin was attached to the back of the camera.
7. The steel wire was then bent into an 'L' shape and attached to the right side of the glasses (frame) using tape.
8. The wires for the battery were extended and the 9V battery was attached to the left side of the glasses to ensure that the glasses
are not unevenly balanced.
8. Using the haytham software, the position of the camera was checked to ensure the camera feed is filming the entire eye.
It was found that the best position of the eye is below the glasses so it doesn't obstruct the user's vision. 
8. The scene camera was mounted to the right of side of the glasses using sticky tape.

Finger detection
1. Scene camera feed is colour thresholded to check for skin colour or latex glove. For checking the skin colour the image is 
converted to Ycc colour space. When checking for the latex glove the image is converted to HSV space.
2. Blobs are cleaned up using morphological operations.
3. The largest blob in the image is defined as the hand and all other blobs are discarded. If the blob is not above a certain area
it is treated as noise and also removed from the image.
4. The convexity hull of the blob is calculated using openCV's library.
INSERT IMAGE HERE SHOWING CONVEXITY DEFECTS
5. End and start points of the convexity defects are located.
6. If the location of the end and start points satisfy the custom heuristic equation, the defect is defined as a finger.
7. Each defect is checked and the number of fingers is therefore the total defects that satisfy the equation.

Hand movement
1. Centroid of hand contour is determined and initial boundary box is set (20x20 pixels)
2. If centroid doesn't move outside of boundary box for 1.5 seconds it, the current position of the hand is set as the reference.
3. A new boundary box is set (60x60 pixels).
4. If the centroid of the hand moves outside of the box it is classified as a movement.
5. If the centroid moves above the box it is classified as up, below the box is down, left of the box is left and right of the box is right.
6. All the while this is happening, the program is sampling the number of fingers shown and averaging the result. This helps to eliminate
false number of fingers identification. When a movement has happened, the averaged number of fingers is set as the final result.

Client
1. A TCP connection was set up between Haytham and the client program.
2. The program controls an Arduino, patrol robot and computer.
3. Arduino connects to the program via serial connection.
4. Arduino has 3 pins connected to 3 leds on a breadboard.
5. The hand movement 2 Up turns on all the leds and 4 Left turns them all off.
6. Program controls computer using sendMessage function.
7. The hand movement 2 Up minimises all applications open and 4 Left reminimises them all.
8. The patrol robot connects to the computer using an ethernet cable.
9. The number of fingers determines the magnitude of movement and the hand movement controls the direction of movement.
10. e.g. 2Up will move the robot forward with a magnitude of 2. 3Left will rotate the robot anti-clockwise.

% ---------------------------------------------------------------
\subsection{Gaze Tracking Glasses}
%Maybe Diako should take care of this subsection

\subsection{Visual Markers Recognition}

This is a test Figure \ref{visualMarker}

\begin{figure}[tp]
 \fitbitmap{figures/visualMarker.jpg}
 \caption{\textbf{Visual Marker Recognition.} Bla bla bla.}
 % the label is used to put refrences to this figure in the text
 \label{visualMarker}
\end{figure}


\subsection{Hand Gesture Recognition}

Figure \ref{hand}

\begin{figure}[tp]
 \fitbitmap{figures/hand.jpg}
 \caption{\textbf{Hand Pose Recognition Through Scene Camera.} Bla bla.}
 % the label is used to put refrences to this figure in the text
 \label{hand}
\end{figure}

\subsection{Putting it all together}
Figure \ref{systemDiagram}

\begin{figure}[tp]
 \fitbitmap{figures/systemDiagram.jpg}
 \caption{\textbf{System Diagram.} Bla bla.}
 % the label is used to put refrences to this figure in the text
 \label{systemDiagram}
\end{figure}


\section{Application}
% Diako, I was wondering if we can really refer to our trials with the system as a ``pilot study''. What do you think?
We carried out a small pilot study to test the functionality and performance of the system. We decided to test the
system in a environment where 3 ``smart'' objects could be controlled by the system simultaneously: a computer, a set of
leds in a bread board and a robot. The hand gesture recognition module could recognize 5 different states of the hands
as defined by the number of fingers being held up: 1, 2, 3, 4 and 5. A gesture was defined as one of these 5 states plus
one of four spatial directions: up, down, left and right.

The infrared leds could just be turned on and off. Two fingers being held up and an upward movement would turn on the
leds. Four fingers being hold up and a movement to the right would turn them off.

The same hand gestures were used to control the computer. The gestures were mapped to minimizing the current window on
display in the computer GUI and to bring them back up. This was done just as a proof of concept, any other type of
gestures associated to different control commands could be envisioned and implemented.

The robotic control was the most elaborated one. The robot could be made to move forward, backward, right and left. The
numbers of fingers being held up indicated, either the speed for forward and backward movements or the amount of turn to
be made for right and left movements.

The hand gestures could be done with bare hands, but we notice that in environments where the color of the walls could
resemble the skin hue, hand gesture recognition performance would suffer. Using a glove with a distinctive color, not 
present in the rest of the environment enhanced hand recognition performance.

This manuscript's associated video \footnote{\url{http://www.youtube.com/watch?v=tBOfvZboGfk}} provides a good visual
overview of the system at work and how it is being used by two different users to interact with a computer, a breadboard
with a set of light emitting diodes and with a robot.


% ---------------------------------------------------------------
\section{Discussion and Conclusion}
In this work we have shown how to interact with objects in the environment through an innovative combination of gaze and
hand gestures using a set of gaze tracking glasses and a hand gesture recognition module. The method is easily
extensible to multiple objects in the environment and a wide array of hand gestures.

The low-cost headmounted eye tracker used and the gaze estimation algorithms employed do not compensate for parallax
error, i.e. the inability to differentiate between the working plane and the calibration plane
\cite{mardanbegi2012parallax}. This limits the ability to alternating interaction with objects at a distance and objects
up close. Nonetheless, since the scene camera used in the glasses is relatively close to the eye being tracked, see Figure
\ref{gazeTrackingGlasses}, the parallax error was minimized. Furthermore, we notice that during the calibration, using
calibration points situated at different distances (from 1 to 10 meters) would find a compromise between objects
far away and objects up close and would generate good gaze estimation for all sort of distances. We noticed that
gaze estimation accuracy was never an issue for our system. Only over time, if the glasses would move slightly from
their position during calibration, due to sweat on the skin and the glasses sliding slightly, would gaze
estimation degrade marginally.


We did notice problems with the skin detection algorithms when the hand was position within the field of view of the
scene camera. This was markedly noticeable, when the colors of the background were similar to the skin color. Usage of
more sophisticated skin detection algorithms could help to solve this issue.


An important issue of the system was the fact that the user wearing the glasses did not have any sort of feedback in
terms of where within the field of view of the scene camera the hand was placed when it was about to initiate a hand
gesture. This was due to the lack of a display on the glasses to provided visual feedback in terms of how the hand is
positioned within the field of view of the scene camera. We implemented an auditory feedback signal to indicate
that the system had found the hand within the field of view of the scene camera and it was therefore ready to receive a
gesture. We found it that this helped the user but still did not provide real time feedback to carry out small
corrections of hand positioning for proper positioning within the field of view of the scene camera. This issue was due
to the usage of an scene camera with a relatively narrow field of view. Using an scene camera with a wider field of view
should prevent the need of feedback for proper hand positioning since the hand would always fall within the field of
view of the scene camera as long as the arm was stretched in front of the user.


Further work should strive to carry out an extensive quantitative analysis of the performance of the system within a
large user study. More shopisticated hand gestures can also be envisioned. However, complex gaze gestures generate a
cognitive and physiological load on the user. Cognitively it is difficult for users to remember a large set of complex
gestures, and physiologically it is tiring and challenging to complete them. Finding the right trade-off between simple
and complex hand gestures is therefore paramount to successfully use hand gestures as a control input device.



%How to insert a figure (image):
\begin{figure}[tp]
 %there's a figure in the same directory called "graph1.jpg"
 \fitbitmap{figures/screenGazeTracker.jpg}
 \caption{caption...}
 % the label is used to put refrences to this figure in the text
 \label{fig:fig1}
\end{figure}


%\begin{table}[tbp]
% \caption{Some numbers that could be experimental data.}
% \label{tab:tab1}
% \begin{tabular}{lcc}\hline
%           & \multicolumn{2}{c}{Factor 2} \\ \cline{2-3}
% Factor 1  & Condition A  & Condition B   \\ \hline
% First     & 586 (231)    & 649 (255)     \\
%           &    2.2       &    7.5        \\
% Second    & 590 (195)    & 623 (231)     \\
%           &    2.8       &    2.5        \\ \hline
% \end{tabular}
%\end{table}


\section{Discussion}
%assumes that there is a biblio.bib file in the same directory
\bibliography{biblio,library}
\end{document}

% Bolt, R.A. 1982. Eyes at the Interface. In Proc. of Human Factors in Computer Systems Conference, 360-362.
% Ware, C. and Mikaelian, H.T. 1987. An evaluation of an eye tracker as a device for computer input. In Proc. of the ACM CHI + GI-87 Human Factors in Computing Systems Conference, 183-188.
%MacKenzie, I. S., & Zhang, X. (2008). Eye typing using word and letter prediction and a �xation algorithm. In ETRA ’08: Proceedings of the 2008 symposium on Eye tracking research & applications
% Isokoski, P. 2000. Text Input Methods for Eye Trackers Using Off-Screen Targets. In Proceedings of the ACM symposium on Eye tracking research & applications ETRA '00. ACM Press.