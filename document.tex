%\documentclass{article}
\documentclass[10pt,a4paper]{article}
\usepackage{fancyhdr}
\usepackage[margin=2.5cm]{geometry}
	\pagestyle{fancyplain}
% header on first page
		\fancyhf{}
			\lhead{\fancyplain{Proceedings}}
			\chead{}
			\rhead{\fancyplain{ECEM 2013}}
			\fancyfoot{} %please leave the footer blank
\renewcommand{\headrulewidth}{0pt}
\usepackage{color}
\usepackage{subfigure}
\usepackage{graphicx}
\usepackage{mathptmx}
\usepackage{multirow}
\usepackage[english]{babel}
\linespread{0.95}

\title{Gaze supported hand gesture interaction with Robot in mobile situations} %Fill this in

\date{} %keep date blank
\author{First author(youremailadress@ecem.et)$^1$, Second author$^2$, Third author$^3$ \\ \\ %Fill in information for all authors here. Don't forget to include the email adress for the first author
$^1$University of xxxx \\ 
$^2$University of xxxx \\
$^3$University of xxxx} %Fill in affiliation information for all authors here. 


\begin{document}
\maketitle%\begin{center}

% \end{center} ---------------------------------------------------------------
\textbf{
\begin{center}
Abstract 
\end{center}
}%the initial 200 words from the submission page should be copied and pasted here
%---------------------------------------------------------------
\noindent

The 200 word abstract you submitted at the ECEM 2013 website should be copied and pasted here. The 200 word abstract you
submitted at the ECEM 2013 website should be copied and pasted here. The 200 word abstract you submitted at the ECEM
2013 website should be copied and pasted here. The 200 word abstract you submitted at the ECEM 2013 website should be
copied and pasted here. The 200 word abstract you submitted at the ECEM 2013 website should be copied and pasted here.
The 200 word abstract you submitted at the ECEM 2013 website should be copied and pasted here. The 200 word abstract you
submitted at the ECEM 2013 website should be copied and pasted here. The 200 word abstract you submitted at the ECEM
2013 website should be copied and pasted here. The 200 word abstract you submitted at the ECEM 2013 website should be
copied and pasted here. The 200 word abstract you submitted at the ECEM 2013 website should be copied and pasted here.
The 200 word abstract you submitted at the ECEM 2013 website should be copied and pasted here. The 200 word abstract you
submitted at the ECEM 2013 website should ...


%---------------------------------------------------------------
\textbf{
\begin{center}
Introduction
\end{center}
}

%We should clearify that this paper does not contribute in the hand gesture recognition but it uses the gesture and gaze in
% a mobile situation for controlling a robot.

In this work we present a head mounted eye tracker and a video-based hand gesture recognition engine that when
combined can be used to interact with objects in the environment.

I low-cost headmounted eye tracker can be built using off-the-shelf components.  The system consists of safety glasses,
batteries, wireless eye camera and the wireless scene or field of view camera.

The wireless eye camera and the associated software monitors the position of the pupil on the image and an infrared
reflection in the iris. The vector difference between the pupil center and the glint reflection in the iris can be
tracked during a calibration procedure to build a model of the user's gaze.

The calibration procedure consists on the user looking at a number of points on the environment and marking them on
the scene camera video stream while the user fixates on them.  Once a calibration procedure is completed, the gaze
estimation algorithm is able to determining the point of regard of the user in the environment.

We use the open-source Haytham \cite{} project developed by Diako Mardanbegui at the  IT University of Copenhagen for
mobile gaze estimation  and extended it with a hand gesture recognition module that enables specific interaction with objects
in the environment through gaze and hand gestures.

The hand gesture recognition module is able to detect a hand in front of the scene camera and the number of fingers  it
is holding up. We define the gesture as holding the hand  with a preset number of fingers holded up  for a predefined
dwell time and moving it in a particular direction: up, down, left  or right. The hand gesture recognition worked well 
for natural skin color, but using a latex glove of a given color improves performance.

The system recognizes objects in the environment by detecting binary associated to them through the scene camera.

To interact with an object in the environment, the user of the glasses needs to first look at the binary beacon that
identifies the object and then performing the gesture. In this way, only that particular object in the environment 
will react to the hand gesture.


For demonstration purposes in the associated video \cite{}, the system is used to interact with a computer, a breadboard
with a set of light emitting diodes and with a robot

%---------------------------------------------------------------
\textbf{
\begin{center}
Method
\end{center}
}

% ---------------------------------------------------------------
\noindent The method section of the extended summary should include information about participants, apparatus,
materials, procedure and analyses.
At a maximum you may use two full pages. Use the predefined format of this template and do not change the format in any
respect. For example, do not change the margins, font, font size, spacing between lines for either headers or text.
All sections (method, results and conclusion) need to be considered. But you are free to choose how much text you want
to include in each of these sections. If you are submitting a theoretical paper you may change the headers of the
sections method and results to something more appropriate.
If you need to include references in the extended summary they should follow APA standards and be included in full in
the text in square brackets, e.g., [Author, X. (2000). My amazing paper. \emph{Journal of Amazingology}, 3(21), pp.
1-2]. The extended summary is the basis for the reviewers. Notification of abstract acceptance for talks will be sent April 15
2013. We cannot guarantee that we will respect the authors' preferences for oral or poster presentation. Some talks may
be offered as poster presentations instead.



\textbf{Related Work}
%Related work on mobile gaze-based interaction, and hand gesture interaction with robot
\textbf{Combining gaze and gesture}
\textbf{Gaze for Interaction}
%methods and limitations
\textbf{Gesture for Interaction}
%
\textbf{Gaze enhanced interaction }
%Pehaps different methods or modes that has been studied in our case
\texbf{Applications}
%Pehaps here we  can talk about different applications and then focued on our example in the next section
\textbf{Experimental example}
\textbf{}
%Hardware and softwares and the environment of our application

\textbf{}
%Different gestures, defining the method and steps








%---------------------------------------------------------------
\textbf{
\begin{center}
Results
\end{center}
}
%---------------------------------------------------------------
\noindent
The results section of the extended summary should include the results of your study. You may insert tables and figures. Table 1 illustrates the style of a table. The factors, conditions and values are fictive. The table caption should be located above the table. Do not insert more than one table and more than one figure.

\begin{table}[\!ht]
	\centering
	\caption{This is a caption for the table}
	\vskip 0.12in
		\begin{tabular}{lcc}
			\hline
				& \multicolumn{2}{c}{Factor 2} \\
					\cline{2-3}
						Factor 1 & Condition A & Condition B \\
						\hline
						First  & 23(11) & 49(10) \\
						Second   & 22(12)  & 12(15) \\
						\hline
		\end{tabular}
	\label{ECEMtableTemplate}
\end{table}

For figures, use PNG, PSD, TIFF, BMP or JPG with a minimum resolution of 300 dpi. Figure 1 illustrates the style of a figure. The figure caption should be located below the figure.

\begin{figure}[h]
	\centering					\includegraphics[width=1.00\textwidth]{C:/ECEM.png}%Add file path and filenanme for your figure here
	\caption{This is a caption for the figure}
	\label{fig:ECEMfigureTemplate}
\end{figure}

%---------------------------------------------------------------
\textbf{
\begin{center}
Conclusion
\end{center}
}
%---------------------------------------------------------------
\noindent
The conclusion section should include your conclusions, e.g., your interpretation of the results in respect to theoretical and/or practical implications. 

\end{document}